% !TEX TS-program = pdflatex
% !TEX encoding = UTF-8 Unicode

% This is a simple template for a LaTeX document using the "article" class.
% See "book", "report", "letter" for other types of document.

\documentclass[14pt]{article} % use larger type; default would be 10pt

\usepackage[utf8]{inputenc} % set input encoding (not needed with XeLaTeX)

%%% Examples of Article customizations
% These packages are optional, depending whether you want the features they provide.
% See the LaTeX Companion or other references for full information.

%%% PAGE DIMENSIONS
\usepackage{geometry} % to change the page dimensions
\geometry{a4paper} % or letterpaper (US) or a5paper or....
% \geometry{margin=2in} % for example, change the margins to 2 inches all round
% \geometry{landscape} % set up the page for landscape
%   read geometry.pdf for detailed page layout information

\usepackage{graphicx} % support the \includegraphics command and options

% \usepackage[parfill]{parskip} % Activate to begin paragraphs with an empty line rather than an indent

%%% PACKAGES
\usepackage{booktabs} % for much better looking tables
\usepackage{array} % for better arrays (eg matrices) in maths
\usepackage{paralist} % very flexible & customisable lists (eg. enumerate/itemize, etc.)
\usepackage{verbatim} % adds environment for commenting out blocks of text & for better verbatim
\usepackage{subfig} % make it possible to include more than one captioned figure/table in a single float
% These packages are all incorporated in the memoir class to one degree or another...

%%% HEADERS & FOOTERS
\usepackage{fancyhdr} % This should be set AFTER setting up the page geometry
\pagestyle{fancy} % options: empty , plain , fancy
\renewcommand{\headrulewidth}{0pt} % customise the layout...
\lhead{}\chead{}\rhead{}
\lfoot{}\cfoot{\thepage}\rfoot{}

%%% SECTION TITLE APPEARANCE
\usepackage{sectsty}
\allsectionsfont{\sffamily\mdseries\upshape} % (See the fntguide.pdf for font help)
% (This matches ConTeXt defaults)

%%% ToC (table of contents) APPEARANCE
\usepackage[nottoc,notlof,notlot]{tocbibind} % Put the bibliography in the ToC
\usepackage[titles,subfigure]{tocloft} % Alter the style of the Table of Contents
\renewcommand{\cftsecfont}{\rmfamily\mdseries\upshape}
\renewcommand{\cftsecpagefont}{\rmfamily\mdseries\upshape} % No bold!

%%% END Article customizations

%%% The "real" document content comes below...

\title{A.I. Systems Integration Framework}

\author{
	Vikrant Varma\\
	11CS30040
 \and 
	Mihir Rege\\
	11CS30026
}
%\date{} % Activate to display a given date or no date (if empty),
         % otherwise the current date is printed 

\begin{document}
\maketitle

\section{Motivation}

The core idea of A.I. systems integration is making individual software components, such as speech synthesizers, interoperable with other components, such as common sense knowledgebases, in order to create larger, broader and more capable A.I. systems. The main methods that have been proposed for integration are message routing, or communication protocols that the software components use to communicate with each other, often through a middleware blackboard system.

Most artificial intelligence systems involve some sort of integrated technologies, for example the integration of speech synthesis technologies with that of speech recognition. However, in recent years there has been an increasing discussion on the importance of systems integration as a field in its own right. A reason for the recent attention A.I. integration is attracting is that there have already been created a number of (relatively) simple A.I. systems for specific problem domains (such as computer vision, speech synthesis, etc.), and that integrating what's already available is a more logical approach to broader A.I. than building monolithic systems from scratch.

\section{Objectives}

\begin{enumerate}
	\item Study and compare existing approaches to AI systems integration
	\item Develop the theoretical model for our framework.
		\begin{itemize}
			\item Understand how AI components could be decomposed into modules, and the relationships between these modules
			\item Develop a communication protocol between such components
			\item Develop a way of specifiying the semantics of such communication
		\end{itemize}
	\item Write a working implementation of the framework, using it to guide the formulation of the theory and test assumptions. Goals for the implementation are in the following order of importance.
		\begin{itemize}
			\item Easy for programmers to use
			\item Applicable to a broad class of real world problems
			\item Modular and extensible
			\item Distributed architecture, to utilize the computing power of more than one computer
		\end{itemize}

\end{enumerate}

\section{Existing Approaches}

\begin {enumerate}
	\item OpenAIR
	\item CORBA
	\item CALO
\end{enumerate}
	
\end{document}
